\chapter{Pendahuluan}
\section{Latar Belakang}

Riset pemrosesan bahasa natural untuk bahasa Indonesia saat ini terbilang sedikit.
Bahkan, masih banyak area riset yang belum tersentuh seperti contohnya
\textit{combinatory categorial grammar} (CCG).
CCG merupakan sebuah formalisme tatabahasa yang dapat dimanfaatkan untuk membangun CCG \textit{parser}.
CCG \textit{parser} dapat digunakan untuk mendapatkan berbagai macam informasi dari suatu kalimat.
Sebagai contoh, CCG \textit{parser} dapat mem-\textit{parse} kalimat
"sebutkan negara-negara yang bertetangga dengan Indonesia" ke dalam bentuk formal yang dapat dipahami
oleh komputer yaitu $\lambda{(x). negara(x) \land bertetangga(x, Indonesia)}$.
Bentuk formal tersebut merupakan \textit{lambda calculus}
(umumnya bentuk formal yang digunakan adalah \textit{combinatory logic}) yang kemudian dapat diproses
oleh komputer strukturnya dan akan mendapatkan \textit{query} untuk mencari $x$ yang berupa
suatu negara dan memiliki ketetanggaan dengan Indonesia.

CCG \textit{parser} membutuhkan CCG \textit{lexicon} (atau dikenal juga sebagai CCG \textit{supertag})
untuk dapat melakukan tugasnya. Sejauh ini belum ditemukan adanya riset mengenai CCG untuk
bahasa Indonesia termasuk pada tahap awalnya yaitu pembentukan CCG \textit{supertag}.
Demikian itu, tugas akhir ini diharapkan dapat menjadi inisiator riset untuk area CCG dalam
pemrosesan bahasa alami yaitu dengan membangun perangkat lunak untuk menghasilkan CCG \textit{supertag}
bahasa Indonesia. Proses yang menghasilkan \textit{supertag} tersebut bernama \textit{supertagging}
adapun perangkat lunaknya bernama \textit{supertagger}.

\section{Perumusan Masalah}
Rumusan masalah yang akan diangkat yaitu:
\begin{enumerate}
    \item Mengapa CCG \textit{supertagger} diperlukan?
    \item Apa saja yang harus dipersiapkan untuk membangun CCG \textit{supertagger}?
    \item Bagaimana proses membangun CCG \textit{supertagger}?
\end{enumerate}
\section{Tujuan}
Tujuan yang diharapkan dapat tercapai oleh tugas akhir ini yaitu:
\begin{enumerate}
    \item Mengenalkan alternatif metode yang dapat digunakan dalam pemrosesan bahasa alami untuk
      bahasa Indonesia.
    \item Merilis CCG \textit{supertagger} pertama untuk bahasa Indonesia.
    \item Membuka peluang riset untuk CCG \textit{parser} bahasa Indonesia.
\end{enumerate}
\section{Batasan Masalah}
Hipotesis dari tugas akhir ini yaitu:
\begin{enumerate}
    \item Memberikan label CCG untuk proses \textit{learning} merupakan permasalahan utama dari tugas akhir
        ini.
    \item Supertagger yang akan dibangun kemungkinan besar memiliki akurasi yang cenderung rendah.
    \item CCG \textit{lexicon} sudah dapat digunakan oleh CCG \textit{parser} (apabila ada).
\end{enumerate}

\section{Rencana Kegiatan}
Rencana kegiatan yang akan dilakukan adalah sebagai berikut:
\begin{itemize}
    \item Studi literatur
    \item Studi \textit{tools} yang tersedia
    \item Studi bahasa pemrograman yang akan digunakan
    \item Perancangan sistem \textit{supertagger}
    \item Membangun \textit{supertagger}
    \item Memeriksa hasil
\end{itemize}

\section{Jadwal Kegiatan}
Laporan proposal ini akan dijadwalkan sesuai dengan tabel \ref{ScheduleTable}. 
\begin{table}[h!]
  \centering
    \caption{Jadwal kegiatan proposal tugas akhir.}
  \label{ScheduleTable}
  \begin{tabular}{|c|m{2.5cm}|m{0.01cm}|m{0.01cm}|m{0.01cm}|m{0.01cm}|m{0.01cm}|m{0.01cm}|m{0.01cm}|m{0.01cm}|m{0.01cm}|m{0.01cm}|m{0.01cm}|m{0.01cm}|m{0.01cm}|m{0.01cm}|m{0.01cm}|m{0.01cm}|m{0.01cm}|m{0.01cm}|m{0.01cm}|m{0.01cm}|m{0.01cm}|m{0.01cm}|m{0.01cm}|m{0.01cm}|}
    \hline
    \multirow{2}{*}{\textbf{No}} & \multirow{2}{*}{\textbf{Kegiatan}} & \multicolumn{24}{|c|}{\textbf{Bulan ke-}} \\
    \hhline{~~------------------------}
    {} & {} & \multicolumn{4}{|c|}{\textbf{1}} & \multicolumn{4}{|c|}{\textbf{2}} & \multicolumn{4}{|c|}{\textbf{3}} & \multicolumn{4}{|c|}{\textbf{4}} & \multicolumn{4}{|c|}{\textbf{5}} & \multicolumn{4}{|c|}{\textbf{6}}\\
    \hline
    1 & Studi Literatur & \cellcolor{blue!25} & \cellcolor{blue!25} & \cellcolor{blue!25} & \cellcolor{blue!25}& \cellcolor{blue!25} & \cellcolor{blue!25} & \cellcolor{blue!25} & \cellcolor{blue!25}& \cellcolor{blue!25} & \cellcolor{blue!25} & \cellcolor{blue!25} & \cellcolor{blue!25}& \cellcolor{blue!25} & \cellcolor{blue!25} & \cellcolor{blue!25} & \cellcolor{blue!25}& \cellcolor{blue!25} & \cellcolor{blue!25} & \cellcolor{blue!25} & \cellcolor{blue!25}& \cellcolor{blue!25} & \cellcolor{blue!25} & \cellcolor{blue!25} & \cellcolor{blue!25}\\
    \hline
    2 & Studi \textit{Tools} yang Tersedia & \cellcolor{blue!25} & \cellcolor{blue!25} & \cellcolor{blue!25} & \cellcolor{blue!25} &  \cellcolor{blue!25} &  \cellcolor{blue!25} &  \cellcolor{blue!25} &  \cellcolor{blue!25} & {} & {} & {} & {}& {} & {} & {} & {}& {} & {} & {} & {}& {} & {} & {} & {}\\
    \hline
    3 & Studi Bahasa Pemrograman & {} & {} & {} & {} & \cellcolor{blue!25} & \cellcolor{blue!25} & \cellcolor{blue!25} & \cellcolor{blue!25} & {} & {} & {} & {}& {} & {} & {} & {}& {} & {} & {} & {}& {} & {} & {} & {}\\
    \hline
    4 & Pengumpulan Data & {} & {} & {} & {} & \cellcolor{blue!25} & \cellcolor{blue!25} & \cellcolor{blue!25} & \cellcolor{blue!25} & \cellcolor{blue!25} & \cellcolor{blue!25} & \cellcolor{blue!25} & \cellcolor{blue!25} & {} & {} & {} & {}& {} & {} & {} & {} & {} & {} & {} & {}\\
    \hline
    5 & Analisis dan Perancangan Sistem &  {} & {} & {} & {}  & \cellcolor{blue!25} & \cellcolor{blue!25} & \cellcolor{blue!25} & \cellcolor{blue!25} & \cellcolor{blue!25} & \cellcolor{blue!25} & \cellcolor{blue!25} & \cellcolor{blue!25} & {} & {} & {} & {}& {} & {} & {} & {}& {} & {} & {} & {}\\
    \hline
    6 & Implementasi Sistem &  {} & {} & {} & {} & {} & {} & {} & {}& \cellcolor{blue!25} & \cellcolor{blue!25} & \cellcolor{blue!25} & \cellcolor{blue!25} & \cellcolor{blue!25} & \cellcolor{blue!25} & \cellcolor{blue!25} & \cellcolor{blue!25} & {} & {} & {} & {}& {} & {} & {} & {}\\
    \hline
    7 & Analisa Hasil Implementasi &  {} & {} & {} & {} & {} & {} & {} & {}& {} & {} & {} & {} & \cellcolor{blue!25} & \cellcolor{blue!25} & \cellcolor{blue!25} & \cellcolor{blue!25} & \cellcolor{blue!25} & \cellcolor{blue!25} & \cellcolor{blue!25} & \cellcolor{blue!25} & {} & {} & {} & {}\\
    \hline
    8 & Penulisan Laporan & {} & {} & {} & {} & \cellcolor{blue!25} & \cellcolor{blue!25} & \cellcolor{blue!25} & \cellcolor{blue!25}& \cellcolor{blue!25} & \cellcolor{blue!25} & \cellcolor{blue!25} & \cellcolor{blue!25}& \cellcolor{blue!25} & \cellcolor{blue!25} & \cellcolor{blue!25} & \cellcolor{blue!25}& \cellcolor{blue!25} & \cellcolor{blue!25} & \cellcolor{blue!25} & \cellcolor{blue!25}& \cellcolor{blue!25} & \cellcolor{blue!25} & \cellcolor{blue!25} & \cellcolor{blue!25}\\
    \hline
  \end{tabular}
\end{table}
