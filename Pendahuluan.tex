\chapter{Pendahuluan}
\section{Latar Belakang}

Riset pemrosesan bahasa natural untuk bahasa Indonesia saat ini terbilang sangat sedikit.
Bahkan, masih banyak area riset yang belum tersentuh seperti contohnya
\textit{combinatory categorial grammar} (CCG).

% \section{Perumusan Masalah}
% Berikut rumusan masalah yang ingin saya angkat adalah
% \begin{enumerate}
%     \item Mengapa ini terjadi?
%     \item Bagaimana proses kejadiannya?
%     \item Apa saja yang dipengaruhinya?
% \end{enumerate}
% \section{Tujuan}
% Berikut adalah tujuan yang ingin dicapai pada penulisan proposal/TA.
% \begin{enumerate}
%     \item Untuk mengetahui mengapa ini terjadi;
%     \item Untuk mempelajari proses kejadian masalah;
%     \item Untuk melihat dampak yang dipengaruhi oleh kejadian ini.
% \end{enumerate}
% \section{Batassan Masalah}
% Hipotesis dari tulisan ini adalah
% \begin{enumerate}
%     \item Masalah timbul karena A;
%     \item Hasil numeriknya menuju $x \rightarrow \infty$
% \end{enumerate}

% \section{Rencana Kegiatan}
% Rencana kegitana yang akan saya lakukan adalah sebagia berikut:
% \begin{itemize}
%     \item Studi literatur
%     \item Memeriksa hasil
% \end{itemize}
% \section{Jadwal Kegiatan}
% The table \ref{Novella} is an example of referenced \LaTeX elements. Laporan proposal ini akan dijadwalkan sesuai dengan tabel yang diberikna berikutnya. 

 
% \begin{table}[h!]
%   \centering
%     \caption{Jadwal kegiatan proposal tugas akhir}
%   \label{Novella}
%   \begin{tabular}{|c|m{2.5cm}|m{0.01cm}|m{0.01cm}|m{0.01cm}|m{0.01cm}|m{0.01cm}|m{0.01cm}|m{0.01cm}|m{0.01cm}|m{0.01cm}|m{0.01cm}|m{0.01cm}|m{0.01cm}|m{0.01cm}|m{0.01cm}|m{0.01cm}|m{0.01cm}|m{0.01cm}|m{0.01cm}|m{0.01cm}|m{0.01cm}|m{0.01cm}|m{0.01cm}|m{0.01cm}|m{0.01cm}|}
%     \hline
%     \multirow{2}{*}{\textbf{No}} & \multirow{2}{*}{\textbf{Kegiatan}} & \multicolumn{24}{|c|}{\textbf{Bulan ke-}} \\
%     \hhline{~~------------------------}
%     {} & {} & \multicolumn{4}{|c|}{\textbf{1}} & \multicolumn{4}{|c|}{\textbf{2}} & \multicolumn{4}{|c|}{\textbf{3}} & \multicolumn{4}{|c|}{\textbf{4}} & \multicolumn{4}{|c|}{\textbf{5}} & \multicolumn{4}{|c|}{\textbf{6}}\\
%     \hline
%     1 & Studi Literatur & \cellcolor{blue!25} & \cellcolor{blue!25} & \cellcolor{blue!25} & \cellcolor{blue!25}& \cellcolor{blue!25} & \cellcolor{blue!25} & \cellcolor{blue!25} & \cellcolor{blue!25}& \cellcolor{blue!25} & \cellcolor{blue!25} & \cellcolor{blue!25} & \cellcolor{blue!25}& \cellcolor{blue!25} & \cellcolor{blue!25} & \cellcolor{blue!25} & \cellcolor{blue!25}& \cellcolor{blue!25} & \cellcolor{blue!25} & \cellcolor{blue!25} & \cellcolor{blue!25}& \cellcolor{blue!25} & \cellcolor{blue!25} & \cellcolor{blue!25} & \cellcolor{blue!25}\\
%     \hline
%     2 & Pengumpulan Data & \cellcolor{blue!25} & \cellcolor{blue!25} & \cellcolor{blue!25} & \cellcolor{blue!25} & {} & {} & {} & {} & {} & {} & {} & {}& {} & {} & {} & {}& {} & {} & {} & {}& {} & {} & {} & {}\\
%     \hline
%     3 & Analisis dan Perancangan Sistem &  {} & {} & {} & {}  & \cellcolor{blue!25} & \cellcolor{blue!25} & \cellcolor{blue!25} & \cellcolor{blue!25} & \cellcolor{blue!25} & \cellcolor{blue!25} & \cellcolor{blue!25} & \cellcolor{blue!25} & {} & {} & {} & {}& {} & {} & {} & {}& {} & {} & {} & {}\\
%     \hline
%     4 & Implementasi Sistem &  {} & {} & {} & {} & {} & {} & {} & {}& \cellcolor{blue!25} & \cellcolor{blue!25} & \cellcolor{blue!25} & \cellcolor{blue!25} & \cellcolor{blue!25} & \cellcolor{blue!25} & \cellcolor{blue!25} & \cellcolor{blue!25} & {} & {} & {} & {}& {} & {} & {} & {}\\
%     \hline
%     5 & Analisa Hasil Implementasi &  {} & {} & {} & {} & {} & {} & {} & {}& {} & {} & {} & {} & \cellcolor{blue!25} & \cellcolor{blue!25} & \cellcolor{blue!25} & \cellcolor{blue!25} & \cellcolor{blue!25} & \cellcolor{blue!25} & \cellcolor{blue!25} & \cellcolor{blue!25} & {} & {} & {} & {}\\
%     \hline
%     6 & Penulisan Laporan & {} & {} & {} & {} & \cellcolor{blue!25} & \cellcolor{blue!25} & \cellcolor{blue!25} & \cellcolor{blue!25}& \cellcolor{blue!25} & \cellcolor{blue!25} & \cellcolor{blue!25} & \cellcolor{blue!25}& \cellcolor{blue!25} & \cellcolor{blue!25} & \cellcolor{blue!25} & \cellcolor{blue!25}& \cellcolor{blue!25} & \cellcolor{blue!25} & \cellcolor{blue!25} & \cellcolor{blue!25}& \cellcolor{blue!25} & \cellcolor{blue!25} & \cellcolor{blue!25} & \cellcolor{blue!25}\\
%     \hline
%   \end{tabular}

% \end{table}
